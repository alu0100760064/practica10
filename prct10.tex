\documentclass[a4paper,12pt]{article}
\usepackage[utf8]{inputenc} 
\usepackage[spanish]{babel}
\usepackage{graphicx}



\begin{document}
\title{Número $\pi$}
\author{Elizabeth Diez Rodriguez\\
        Tecnicas experimentales~\footnote{Informe sobre el  número $\pi$} 
        }
\date{09/04/2014}
\maketitle



\begin{abstract}
$\pi$ (pi) es la relación entre la longitud de una circunferencia y su diámetro, en geometría euclidiana.
Es un número irracional y una de las constantes matemáticas más importantes. Se emplea frecuentemente en matemáticas, 
física e ingeniería. El valor numérico de $\pi$, truncado a sus primeras cifras, es el siguiente:

    $\pi$ = 3,14159265358979323846  
\end{abstract}


\section{Descrippcion}

\subsection{Valor de $\pi$}
El valor de $\pi$ se ha obtenido con diversas aproximaciones a lo largo de la historia,
siendo una de las constantes matemáticas que más aparece en las ecuaciones de la física, junto con el número e.
Cabe destacar que el cociente entre la longitud de cualquier circunferencia y 
la de su diámetro no es constante en geometrías no euclídeas.

 \cite{pi2}
 
\subsection{El nombre $\pi$}
La notación con la letra griega $\pi$ proviene de la inicial de las palabras de origen griego $\pi$epilyepela 'periferia'
y $\pi$epilyepela 'perímetro' de un círculo,1 notación que fue utilizada primero por William Oughtred (1574-1660) y 
cuyo uso fue propuesto por el matemático galés William Jones2 (1675-1749); aunque fue el matemático Leonhard Euler,
con su obra Introducción al cálculo infinitesimal, de 1748, quien la popularizó. Fue conocida anteriormente como constante
de Ludolph (en honor al matemático Ludolph van Ceulen) o como constante de Arquímedes (que no se debe confundir con el número
de Arquímedes).

En la tabla \ref{table1} se puede ver el total de chicos que hay en las clases 1,2 y 3.

\section{Historia}
\subsection{Antiguo egipto}
El valor aproximado de $\pi$ en las antiguas culturas se remonta a la época del escriba egipcio Ahmes en el año 1800 a. C., 
descrito en el papiro Rhind,3 donde se emplea un valor aproximado de $\pi$ afirmando que el área de un círculo es similar a 
la de un cuadrado cuyo lado es igual al diámetro del círculo disminuido en 1/9; es decir, igual a 8/9 del diámetro.

\cite{pi}

\subsection{Mesopotamia}
Algunos matemáticos mesopotámicos empleaban, en el cálculo de segmentos, valores de $\pi$ igual a 3, a
lcanzando en algunos casos valores más aproximados

    

En la figura \ref{fig.1} se puede ver el logotipo del número $\pi$ que se va a hacer el informe.

\begin{table}[h]
\begin{center}
 \begin{tabular}{|l|c|c|c|}
 \hline
 
clase & chicos & chicas & total \\ \hline
 1 & 15 & 25 & 40 \\ \hline
 2 & 10 & 30 & 40 \\ \hline
 3 & 20 & 2 & 22 \\ \hline
 
  
\end{tabular}
\caption{mi tabla}
\label{table1}
\end{center}
\end{table}


\begin{thebibliography}{00}
 \bibitem{pi}
 http://es.wikipedia.org
 \bibitem{pi2}
 http://webs.adam.es/rllorens/pi.htm
\end{thebibliography}

\begin{figure}[t]
 \includegraphics[scale=0.10]{imagen1.eps}
 \caption{ejemplo de grafica}
 \label{fig1}
\end{figure}


\end{document}

